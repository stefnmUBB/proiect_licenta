\section{Implementation. Technical documentation}


\subsection{C\# programming language. .NET framwework.}

C\# is a high level object-oriented and general-purpose programming language developed by Microsoft along with the .NET platform \cite{csharp}. It has seen use in a wide palette of software applications, such as Windows desktop and web services applications, games and, due to Microsoft's most recent efforts, cross-platform applications. C\# permits writing modern secure and performant software. What makes it stand out from other programming languages is the fact that besides featuring a type safety mechanism, it also allows for manipulating pointer types directly. This aspect is heavily exploited in the current project since pointer access cuts down computation time significantly compared to using native array structures. In addition, C\# features built-in mechanisms sometimes even with syntax support for event driven mechanisms, sequence manipulations via LINQ queries, code attributing, reflection, regular expressions, named value tuple types generating on-the-fly. It is worth mentioning other valuable features like dynamic typing, which makes it feel more like Python, or runtime assembly code generation \cite{csharp70}. 

The .NET framework is the platform that powers C\# applications and its relatives such as Visual Basic, Visual C/C++ or F\#. Many Windows and web applications/services are build upon the .NET framework. Some notable mentions are Winforms\cite{winforms}, WPF and ASP.NET. .NET itself comes in many flavors and versions, such as .NET Framework, .NET Core, .NET Standard. While C\# is theoretically a platform-idependent language as it compiles to an intermediate language (MSIL) of the Common Language Runtime (CLR), which is the foundation of .NET framework, it has well matured on the Windows operating system \cite{csharp70}. Other framework have been built upon different platforms along time, like the Universal Windows Platform (UWP) for Windows 10 devices, including mobile and XBox, .NET Core for Linux and MacOS which supports deploying self-contained apps, and Xamarin for Android, IOS and Windows Mobile devices \cite{csharp70}. Newer developments from Microsoft try to come up with a truly multiplatform solution in the form of the open-source Mono project and the .NET Multi-Platform App UI (MAUI) framework \cite{MAUI}. MAUI is quite a new technology and naturally has its caveats, and the author's discouraging experience with this framework has contributed to the choice of preferring an older, but more stable and intuitive version of Xamarin framework.

\subsection{Xamarin}

Xamarin provides C\# bindings to the native device API. The current use of Xamarin is for Android development, although nowadays it supports a wider variety of platforms. Xamarin.Forms is a UI toolkit that contains wrappers around native UI classes, allowing them to be used in a way which resembles the style of WPF projects. Xamarin.Forms basically makes it possible for the same UI code to run on multiple platforms. It relies on the Extensible Markup Language (XAML) as a clean and fast-forward way to define UI interfaces, without needing knowledge of particular details like the workings of Android graphical interface. However, familiarity with lower level platform-specific technicalities is required for user-defined UI components with custom behavior. Xamarin keeps the UI simple graviting around a Page, Layout, View system \cite{xamarin}.

\subsection{Tensorflow}

Tensorflow is an open-source library dedicated to machine learning and artifial intelligence \cite{Tensorflow}. Although not directly related to the final product, a version of Tensorflow 2.15 has been used for configuring and training the neural network models. For the reason of understanding improvement through self-study, the author has insisted on providing own implementation of neural network layers instead of relying on deployment options like Tensorflow Lite. Moreover, the great popular libraries TensorflowSharp and Tensorflow.NET contain bindings to the Tensorflow library in a way or another. They are by far useful and complex alternatives for machine learning in .NET, but are either outdated, limited by a particular framework version or a level of complexity too high for their actual use case. The current project contains a pure C\# intferation-only AI toolkit targeting .NET Standard 2.0 assuring increased compatibility to satisfy the need of running models inside the application. 
